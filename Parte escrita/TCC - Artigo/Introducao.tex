\newpage
\section{INTRODUÇÃO}

\pagenumbering{arabic}

O basquete é um dos esportes mais populares do mundo,e também,	 uma esporte de equipe, em que o objetivo é atirar uma bola através de uma cesta posicionada horizontalmente, para marcar pontos, com um conjunto de regras. Normalmente, há duas equipes de cinco jogadores jogando em uma quadra retangular marcado, cada lado com uma cesta. Com sua popularidade surgiu o interesses na predição de suas partidas. 

%\cite{Griffiths2010} \cite{Stekler2010}\cite{Papic2009}\cite{Kononenko95}\cite{nba}\cite{juputer}cite{python}

\textit{National Basketball Associativo} (NBA) desde a sua origem tem mais de 60 anos. Durante esta organização, há 30 equipes formadas e divididas em Conferência Leste e Conferência Oeste. Para a temporada regular terá 82 jogos para cada equipe e pós temporada usando um esquema de melhor de sete séries. Portanto, uma estimativa que, haverá pelo menos cerca de 2.300 jogos gerados. Com uma massa de dados e gerada depois de cada jogo da NBA, esses dados existentes nos permitem descobrir dados valioso.

Por isso os sistemas computacionais que utiliza a predição passaram a ser usado, para auxiliar na formação de novas estratégias para aqueles que necessitam de dados mais preciso e confiáveis. A matemática tornaram-se uma parte importante do esporte e muito esforço é dedicado a prever os resultados de eventos esportivo.

Já tendo sido usado em vários processos de tomada de decisão, a predição pode fornecer para os treinadores uma visão sobre o desempenho de suas equipes durante um jogo. Assim podendo simular como uma partida é estudar cenários que podem surgir em diferentes circunstâncias na quadra. Assim ajudará os treinadores a entender como uma equipe pode aumentar suas chances de ganhar, como as habilidades em jogos individuais afetam o desempenho da equipe e qual desempenho pode ser esperado usando diferentes abordagens.


\section{REFERENCIAL TEÓRICO}
A análise computacional é uma maneira objetiva de registrar o desempenho, de modo que os eventos críticos nesse desempenho podem ser quantificados de maneira consistente e confiável. Essa análise permite que o treinador e o gerente avaliem objetivamente o desempenho competitivo e, portanto, melhorem-no \cite{Taylor2004}. 

A precisão e a velocidade das previsões dependerão da seleção manual ou automática adequada dos recursos mais significativos e altamente correlacionados. Kahn avaliou características primárias e empregou o método sugerido por \cite{Purucker1996} para selecionar cinco características finais para predição \cite{Kahn2003}.

As métricas da Associação de Pesquisadores de Basquete Profissional (ABPR) são semelhantes às da sabermetrics, que é uma das primeiras métricas para avaliar o desempenho dos jogadores de beisebol, mas as métricas da ABPR tentam visualizar as estatísticas em termos de desempenho de equipe e não de desempenho individual. Existem muitos fatores incertos para influenciar o resultado, no entanto, a mineração de dados ainda tem seu próprio valor na previsão do resultado na previsão de resultados de jogos de basquete.\cite{Schumaker2010}

\citeonline{Bernard2009} fizeram uma pesquisa sobre a previsão de jogos da NBA usando redes neurais. Autores exploraram subconjuntos obtidos a partir de relações sinal-ruído e opiniões de especialistas para identificar um subconjunto de recursos de entrada para as redes neurais. Os resultados obtidos a partir dessas redes foram comparados com as previsões feitas por vários especialistas no campo do basquete. Após o experimento, o projeto teve 70,33\% de precisão.

Embora o treinamento de um Máquina de vetores de suporte (MVS) leve mais tempo comparado a outros métodos, acredita-se que o algoritmo tenha alta precisão devido à sua alta capacidade de construir limites de decisão complexos e não-lineares. Também é menos propenso a \textit{overfitting} \cite{Han2017}. 

Cao empregou um MVS, um classificador logístico simples uma combinação de algoritmos cujo núcleo é a regressão logística e usa o \textit{LogitBoost} como uma função de regressão simples \cite{Landwehr2005}, e uma rede neural multicamada para prever resultados de basquete.

\citeonline{Witten2017}, no momento no qual deseja-se estimar o valor de uma variável numérica e os atributos do  conjunto de dados também são numéricos, a escolha pela técnica de regressão linear é natural, a mesma vem sendo utilizada por décadas na aplicação de problemas estatísticos, de modo que mesmo quando o conjunto de dados não apresenta uma dependência linear a aplicação do algoritmo serve como um ponto de partida para a utilização de outros algoritmos mais complexos.

\subsection{OBJETIVO GERAL}
O objetivo deste trabalho é a análise e predição de partidas de basquete utilizando dados das partidas e dos jogadores.

\subsection{OBJETIVOS ESPECÍFICOS}
\begin{itemize}
	\item comparar e demonstrar a eficácia para os classificadores utilizados no estado da arte de predição de partidas de basquete.
	\item comparar e demonstrar a eficácia das bases de dados existentes no estado da arte na predição de partidas de basquete.
	\item comparar e demonstrar a eficácia dos métodos seletores de características utilizados no estado da arte de predição de partidas de basquete.
\end{itemize}
