\section{DESENVOLVIMENTO}
\subsection{MATERIAIS}
Os materiais usados no trabalho foram duas bases de dados ambas da NBA Advanced Stats\footnote{National Basketball Associativo} sendo uma da \textit{season} de 2014 a 2018 com 9.840 jogos com os dados armazenados em um arquivo csv, a outra indo da \textit{season} de 2007 a 2019 com 30.000 jogos com os dados armazenados em um banco de dados e sendo acessado através da nba-api PyPI. 

A ferramenta utilizada para o desenvolvimento foi o \textit{JupyterLab}, linguagem de programação {python} e o uso das bibliotecas  pandas, numpy, \textit{sklearn}, \textit{seaborn}, matplotlib.

A NBA advances stats e um site patrocinado pela SAP com o proposito e de manter um registros de toda liga da NBA e facilitar e acesso a essa informações pelas equipes e organizações. A nba-api PyPI e uma API para acesso a www.nba.com, o principal objetivo e mapear e analisar o maior numero posivel de jogos.

O \textit{jupyterlab} é um ambiente de desenvolvimento interativo baseado na \textit{web} para \textit{notebooks}. O \textit{jupyterlab} é facil de configurar e organizar, a interface do usuário para suporta uma ampla variedade de fluxos de trabalho em ciência de dados, computação científica e aprendizado de máquina.O \textit{jupyterlab} é extensível e modular e fácil de adicionar os \textit{plug-ins}, que adicionam novos componentes e se integram aos já existentes\cite{jupyter}.

Python é uma linguagem de programação criada por Guido van Rossum em 1991. Os objetivos do projeto da linguagem eram produtividade e legibilidade, é uma linguagem de alto nível, multi paradigma, suporta o paradigma orientado a objetos, imperativo, funcional e procedural. Possui tipagem dinâmica e uma de suas principais características é permitir a fácil leitura do código e exigir poucas linhas de código se comparado ao mesmo programa em outras linguagens\cite{python}.

O pandas é uma biblioteca de código aberto, licenciada por BSD \footnote{Berkeley Software Distribution}, que fornece estruturas de dados de alto desempenho e fáceis de usar e ferramentas de análise de dados para a linguagem de programação \textit{python}. O pandas ajuda a preencher essa lacuna, permitindo que você execute todo o fluxo de trabalho de análise de dados no \textit{python} sem precisar mudar para uma linguagem\cite{pandas}.

O numPy é uma biblioteca \textit{python} que é usada para realizar cálculos em \textit{arrays} multidimensionais. Fornecendo um grande conjunto de funções e operações que ajudam os programadores a executar facilmente cálculos numéricos.\cite{numpy}

O \textit{scikit learn} é uma biblioteca \textit{python} que é usada para aprendizado de máquina. Ela possui uma variedade de algoritmos incluindo vários algoritmos de classificação, regressão e agrupamento incluindo máquinas de vetores de suporte, florestas aleatórias, \textit{gradient boosting}, \textit{k-means}\cite{scikit}.

O matplotlib é uma biblioteca de plotagem 2D do python, é uma biblioteca que tenta facilitar e facilitar a gerar gráficos, histogramas, espectros de potência, gráficos de barras, gráficos de erros, gráficos de dispersão etc\cite{matplotlib}.

O seaborn é uma biblioteca de visualização de dados Python baseada no matplotlib . Ele fornece uma interface de alto nível para desenhar gráficos estatísticos atraentes e informativos.\cite{seaborn}


\subsection{METODOLOGIA}
Os algoritmos usados para as predições são os de regressão linear, regressão logística, k Nearest Neighbors,  arvore de decisão, floresta aleatória,  Máquinas de Vetores de Suporte.

O algoritmo de regressão linear responsável por modelar uma associação entre uma ou mais variáveis de saída e entrada. O processo de regressão pode ser dividido em duas categorias, as paramétricas, no qual o relacionamento entre as variáveis é conhecido, e não paramétricas onde não existe conhecimento preexistente entre as variáveis. As técnicas de regressão linear procuram a relação entre duas variáveis por meio de uma equação de uma linha reta.
