\section{DESENVOLVIMENTO}

Para a analise sera utilizado duas bases de dados ambas da NBA\footnote{National Basketball Associativo} sendo uma da \textit{season} de 2014 a 2018 com 9.840 jogos com os dados armazenados em um arquivo csv, a outra indo da  \textit{season} de 2007 a 2019 com 30.000 jogos com os dados armazenados em um banco de dados e sendo acessado através da  nba-api  PyPI. 

Os algoritmos que foi utilizados para as predições são os de regressão linear, regressão logística, k Nearest Neighbors,  arvore de decisão, floresta aleatória,  Máquinas de Vetores de Suporte. Todos foram escritos na linguagem \textit{python} com o uso das bibliotecas  pandas, numpy, \textit{sklearn}. Para realizar as analises dos dados foi usada a \textit{seaborn}, para a criação de graficos matplotlib.


O algoritmo de regressão linear responsável por modelar uma associação entre uma ou mais variáveis de saída e entrada. O processo de regressão pode ser dividido em duas categorias, as paramétricas, no qual o relacionamento entre as variáveis é conhecido, e não paramétricas onde não existe conhecimento preexistente entre as variáveis. As técnicas de regressão linear procuram a relação entre duas variáveis por meio de uma equação de uma linha reta.
