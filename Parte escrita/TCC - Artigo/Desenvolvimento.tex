\section{DESENVOLVIMENTO}
\subsection{MATERIAIS}

Os materiais usados no trabalho foram duas bases de dados da NBA Advanced Stats\footnote{National Basketball Associativo}, sendo uma da \textit{season} de 2014 a 2018 com 9.840 jogos com os dados armazenados em um arquivo csv, a outra é da \textit{season} de 2007 a 2019 com 30.000 jogos com os dados armazenados em um banco de dados e sendo acessado através da nba-api PyPI. 

A NBA advances stats um site patrocinado pela SAP com o proposito de manter um registro de toda liga da NBA e facilitar  acesso a essa informação pelas equipes e organizações. A nba-api PyPI e uma API para acesso a www.nba.com, o principal objetivo e mapear e analisar o maior número possível de jogos.

A ferramenta utilizada para o desenvolvimento foi o \textit{JupyterLab}, linguagem de programação {python} e o uso das bibliotecas  pandas, numpy, \textit{sklearn}, \textit{seaborn}, matplotlib.

Python é uma linguagem de programação criada por Guido van Rossum em 1991. Os objetivos do projeto da linguagem eram produtividade e legibilidade, é uma linguagem de alto nível, multi paradigma, suporta o paradigma orientado a objetos, imperativo, funcional e procedural. Possui tipagem dinâmica e uma de suas principais características é permitir a fácil leitura do código e exigir poucas linhas de código se comparado ao mesmo programa em outras linguagens\cite{python}.

O \textit{jupyterlab} é um ambiente de desenvolvimento interativo baseado na \textit{web} . O \textit{jupyterlab} é facil de configurar e organizar, a interface do usuário suporta uma ampla variedade de fluxos de trabalho em ciência de dados, computação científica e aprendizado de máquina.O \textit{jupyterlab} é extensível e modular é fácil de adicionar os \textit{plug-ins}, que se integram aos já existentes\cite{jupyter}.

O pandas é uma biblioteca de código aberto, licenciada por BSD \footnote{Berkeley Software Distribution}, que fornece estruturas de dados de alto desempenho e fáceis de usar e ferramentas de análise de dados para a linguagem de programação \textit{python}. O pandas ajuda a preencher essa lacuna, permitindo que você execute todo o fluxo de trabalho de análise de dados no \textit{python} sem precisar mudar para uma linguagem\cite{pandas}.

O numPy é uma biblioteca \textit{python} que é usada para realizar cálculos em \textit{arrays} multidimensionais. Fornecendo um grande conjunto de funções e operações que ajudam os programadores a executar facilmente cálculos numéricos.\cite{numpy}

O \textit{scikit learn} é uma biblioteca \textit{python} que é usada para aprendizado de máquina. Ela possui uma variedade de algoritmos, incluindo vários algoritmos de classificação, regressão e agrupamento incluindo máquinas de vetores de suporte, florestas aleatórias, \textit{gradient boosting}, \textit{k-means}\cite{scikit}.

O matplotlib é uma biblioteca de plotagem 2D do python, é uma biblioteca que tenta facilitar a gerar gráficos, histogramas, espectros de potência, gráficos de barras, gráficos de erros, gráficos de dispersão etc\cite{matplotlib}.

O seaborn é uma biblioteca de visualização de dados Python baseada no matplotlib . Ele fornece uma interface de alto nível para desenhar gráficos estatísticos atraentes e informativos.\cite{seaborn}


\subsection{METODOLOGIA}
Os algoritmos usados para as predições são os de regressão linear, regressão logística, k-NN\footnote{k-nearest neighbors},  arvore de decisão, floresta aleatória, máquinas de vetores de suporte.

O algoritmo de regressão linear responsável por modelar uma associação entre uma ou mais variáveis de saída e entrada. O processo de regressão pode ser dividido em duas categorias, as paramétricas, no qual o relacionamento entre as variáveis é conhecido, e não paramétricas onde não existe conhecimento preexistente entre as variáveis. As técnicas de regressão linear procuram a relação entre duas variáveis por meio de uma equação linear\cite{Bogoni2019}.

A regressão logística é uma técnica utilizada para a estimação de uma variável de natureza binária, estimando o valor em 0 ou 1, sendo que as variáveis independentes podem ser de natureza categórica ou não. Como na regressão linear é
necessário aplicar pesos onde ajustam-se aos dados de treinamento do algoritmo, porém a regressão logística não procura a melhor reta que se ajuste aos dados, mas sim a melhor curva. A regressão logística calcula uma razão de probabilidade da variável alvo, que posteriormente é convertida em uma variável de base logarítmica, permitindo assim a classificação com base na aproximação de um dos valores\cite{Witten2011}.

O algoritmo k-NN é um método não paramétrico usado para classificação e regressão . Nos dois casos, a entrada consiste nos k exemplos de treinamento a saída depende se k-NN é usado para classificação ou regressão. Na classificação k-NN, a saída é uma associação de classe. Um objeto é classificado pelo voto de pluralidade de seus vizinhos, sendo o objeto atribuído à classe mais comum entre os k vizinhos mais próximos. Os vizinhos são obtidos de um conjunto de objetos para os quais a classe ou o valor da propriedade do objeto é conhecida\cite{KamgarParsi1985}.

O processo de classificação em uma árvore de decisão, acontece de maneira recursiva, de modo que o nó inicial representa o conjunto de dados, em seguida deve ser avaliado se os objetos são da mesma classe, sendo esse o caso o nó é considerado um nó folha, caso contrário um atributo precisa ser usado para dividir os dados. Este processo deve ser executado recursivamente, ele pode ser descontinuado caso faltarem atributos para realizar testes de divisão ou caso todos os registros forem da mesma classe\cite{castro}.

Florestas aleatórias são um grupo de árvores de decisão, nos quais juntos formam uma floresta. Estas árvores são geradas com base em um atributo aleatório que é o responsável pela divisão em cada nó da árvore. A precisão de uma floresta aleatória é determinada de acordo com a força de cada classificador da árvore, e também o nível de dependência entre eles.O melhor modo de atingir essa precisão é mantendo a força dos classificadores e não aumentar a correlação entre eles\cite{castro}.

A técnica de máquinas de vetores de suporte, têm como fundamento o aprendizado em cima da estatística, o algoritmo apresenta ótima performance na utilização de dados de alta dimensionalidade. O mesmo funciona através de um conceito de hiperplano, sendo definido um limite linear neste plano para realizar a classificação, o algoritmo possuí a função de detectar o hiperplano de margem máxima, aquele com a maior margem separação entre as classes, com o objetivo de apresentar menos erros de generalização em relação a margens menores\cite{Tan2009}.

A escrita dos algoritmos para realizar os testes foi iniciada, com a importação das bibliotecas e foi carregada as bases de dados, apos a base de dados ser carregadas em um \textit{dataframe} da biblioteca padas, foi feita a avaliação de ambas as bases é feita a escolha das características que foi usadas para a predição.

A base1\footnote{base de dados de 2014 a 2018 com 9.840 jogos} contendo 34 características sendo elas: 
\begin{table}[htbp]
	\begin{longtable}{|l|l|} \hline 
		Características              & Descrição                                                                                                                                   \\ \hline 
		\textit{team}                & sigla do \textit{time}                                                                                                                               \\ \hline
		\textit{game}                          & id do jogo                                                                                                                                  \\ \hline
		\textit{date}                        & data do jogo                                                                                                                                \\ \hline
		\textit{opponent}                      & sigla do oponente                                                                                                                           \\ \hline
		\textit{winorloss}                    & vitoria e derrota                                                                                                                           \\ \hline
		\textit{team points}                & pontos do time                                                                                                                              \\ \hline
		\textit{opponent points}               & pontos do oponente                                                                                                                          \\ \hline
		\textit{field goals}                   & \begin{tabular}[c]{@{}l@{}}cesta marcada em qualquer \\ arremesso ou toque que não \\ seja lance livre\end{tabular}                         \\ \hline
		\textit{field goals attempted}        & \begin{tabular}[c]{@{}l@{}}tentativa cesta marcada em \\ qualquer arremesso ou toque \\ que não seja lance livre\end{tabular}               \\ \hline
		\textit{X3 point shots}               & cesta de 3 pontos                                                                                                                           \\ \hline
		\textit{X3 point shots attempted}     & tentativa de cesta de 3 pontos                                                                                                              \\ \hline
		\textit{X3 point shots opp}            & cesta de 3 pontos oponente                                                                                                                  \\ \hline
		\textit{X3 point shots attempted} opp & \begin{tabular}[c]{@{}l@{}}tentativa de cesta de 3 pontos\\ oponente\end{tabular}                                                           \\ \hline
		\textit{free throws}                 & arremessos livres                                                                                                                           \\ \hline
		\textit{free throws attempted}       & tentativa de arremessos livres                                                                                                              \\ \hline
		\textit{free throws opp}               & arremessos livres oponente                                                                                                                  \\ \hline
		\textit{free throws attempted opp}   & \begin{tabular}[c]{@{}l@{}}tentativa de arremessos livres\\ oponente\end{tabular}                                                           \\ \hline
		\textit{rebounds}                      & rebotes                                                                                                                                     \\ \hline
		Total \textit{rebounds}               & total de rebotes                                                                                                                            \\ \hline
		\textit{assists}                       & assistência                                                                                                                                 \\ \hline
		\textit{steals}                       & roubada de bola                                                                                                                             \\ \hline
		\textit{blocks}                       & bloqueio de bola                                                                                                                            \\ \hline
		\textit{turnovers}                    & rotatividade                                                                                                                                \\ \hline
		total \textit{fouls}                    & falta total                                                                                                                                 \\ \hline		
	\end{longtable}
\end{table}
\newpage
\begin{table}[htbp]
	\begin{longtable}{|l|l|} \hline 
		\textit{opp field goals}              & \begin{tabular}[c]{@{}l@{}}cesta marcada em qualquer \\ arremesso ou toque que não \\ seja lance livre do oponente\end{tabular}             \\ \hline
		\textit{opp field goals attempted}    & \begin{tabular}[c]{@{}l@{}}tentativa cesta marcada em \\ qualquer arremesso ou toque \\ que não seja lance livre do\\ oponente\end{tabular} \\ \hline
		\textit{opp off rebounds}              & rebotes dos oponentes                                                                                                                       \\ \hline
		opp total \textit{rebounds}           & total de rebote dos oponentes                                                                                                               \\ \hline
		\textit{opp assists}                  & assistência oponente                                                                                                                        \\ \hline
		\textit{opp steals}                   & roubada de bola oponente                                                                                                                    \\ \hline
		\textit{opp blocks}                   & bloqueio de bola oponente                                                                                                                   \\ \hline
		\textit{opp turnovers}                 & rotatividade do oponente                                                                                                                    \\ \hline
		opp total \textit{fouls}              & total de faltas do oponente \\ \hline                                                                                                               
	\end{longtable}
\end{table}

Depois de um analise as características que não foram relevantes foram excluídas da base de dados, sendo retiradas as \textit{team}, \textit{game}, \textit{date}, \textit{home}, \textit{opponent}.

A base2\footnote{base de dados de 2007 a 2019 com 30.000 jogos} contendo 30 características sendo elas:

\begin{table}[htbp]
	\begin{longtable}{|l|l|}
		\hline
		Características                        & Descrição                                                                                                                                         \\ \hline
		\textit{season id}                     & id da temporada                                                                                                                                   \\ \hline
		\textit{team id}                       & id time                                                                                                                                           \\ \hline
		\textit{team abbreviation}             & abreviação do time                                                                                                                                \\ \hline
		\textit{team name}                     & nome do time                                                                                                                                      \\ \hline
		\textit{game id}                       & id do jogo                                                                                                                                        \\ \hline
		\textit{team out}                      & time de fora                                                                                                                                      \\ \hline
		\textit{match up}                      & confronto individua                                                                                                                               \\ \hline
		\textit{gamedate}                      & data do jogo                                                                                                                                      \\ \hline
		\textit{win loss (W/L)}                & vitoria e derrota                                                                                                                                 \\ \hline
		\textit{minutes}                       & tempo do jogo                                                                                                                                     \\ \hline
		\textit{played}                        & numero de jogadas                                                                                                                                 \\ \hline
		\textit{points}                        & pontuação                                                                                                                                         \\ \hline
		\textit{field goals}                   & \begin{tabular}[c]{@{}l@{}}cesta marcada em qualquer \\ arremesso ou toque que não\\ seja lance livre\end{tabular}                                \\ \hline
		\textit{field goals attempted}         & \begin{tabular}[c]{@{}l@{}}tentativa cesta marcada em \\ qualquer arremesso ou toque \\ que não seja lance livre\end{tabular}                     \\ \hline
		
	\end{longtable}
\end{table}
\newpage
\begin{table}[htbp]
	\begin{longtable}{|l|l|}
		\hline
		\textit{field goal percentage}         & \begin{tabular}[c]{@{}l@{}}porcentagem cesta marcada em \\ qualquer arremesso ou toque \\ que não seja lance livre\end{tabular}                   \\ \hline
		\textit{3 point field goals}           & \begin{tabular}[c]{@{}l@{}}cesta  de 3 pontos marcada \\ em qualquer arremesso ou \\ toque que não seja lance livre\end{tabular}                  \\ \hline
		\textit{3 point field goals attempted} & \begin{tabular}[c]{@{}l@{}}tentativa cesta de 3 pontos \\ marcada em qualquer \\ arremesso ou toque que \\ não seja lance livre\end{tabular}      \\ \hline
		\textit{3 point field goal percentage} & \begin{tabular}[c]{@{}l@{}}porcentagem de cesta de 3 pontos \\ marcada em qualquer \\ arremesso ou toque que \\ não seja lance livre\end{tabular} \\ \hline
		\textit{free throws}                   & arremessos livres                                                                                                                                 \\ \hline
		\textit{free throws attempted}         & tentativa de arremessos livres                                                                                                                    \\ \hline
		\textit{free throw percentage}         & porcentagem                                                                                                                                       \\ \hline
		\textit{offensive rebounds}            & rebotes ofensivos                                                                                                                                 \\ \hline
		\textit{defensive rebounds}            & rebotes defensivos                                                                                                                                \\ \hline
		\textit{rebounds}                      & rebotes                                                                                                                                           \\ \hline
		\textit{assists}                       & assistência                                                                                                                                       \\ \hline
		\textit{steals}                        & roubada de bola                                                                                                                                   \\ \hline
		\textit{blocks}                        & bloqueio de bola                                                                                                                                  \\ \hline
		\textit{turnovers}                     & rotatividade                                                                                                                                      \\ \hline
		\textit{fouls}                         & faltas                                                                                                                                            \\ \hline
		\textit{plus minus}                    & minutos extra    \\ \hline                                                                       
	\end{longtable}
\end{table}

Depois de um analise as características que não foram relevantes foram excluídas da base de dados, sendo retiradas as \textit{season id}, \textit{team id}, \textit{team abbreviation}, \textit{team name}, \textit{game id}, \textit{team out}, \textit{match up}, \textit{gamedate}.
 


 


Após a retirada das características que não serão usadas foi feito um processamento nos dados, transformando as colunas \textit{(W/L)win loss} e \textit{winorloss} que continham os dados de vitoria como "W" e derrota como "L", as linha contendo "W" foi convertida para "1" as com "L" para "0". Foi verificado a ocorrência de dados faltantes na base de dados, e as lacunas foram preenchidas com a média dos dados da coluna.

Com as bases de dados preparada foi feito uma divisão, sendo em duas partes, uma para teste e outra para treino com a base de teste contendo 30 \% dos dados e a base treino contendo os outros 70 \%. Com a base separada em quatro vetores sendo os vetores x\_treino, x\_teste, y\_treino, y\_teste.

Logo após a divisão dos vetores foi feita a instanciação do algoritmos utilizados, e foi chamada a função de treino do algoritmo. Após o treino ser realizado foi feita a chamada do função de predição. Para mostrar os dados foi realizado um plote de um gráfico contendo os dados reais e o que foi previsto, também foi realizado exibição das métricas de erro do algorítimo. A seguir foi realizada a predição usando o método do \textit{Cross Validation} para ver se haveria melhoria nos dados.

\subsection{RESULTADOS} 

O algoritmo de regressão linear com a base1\footnote[4]{base de dados de 2014 a 2018 com 9.840 jogo} teve uma acurácia de 66.6\% veja o grafico de relação entre os dados reais e os previstos. Após aplicar \textit{Cross Validation} essa taxa teve um aumento de 2.3\%.
\begin{figure}[htbp]
	\begin{center}
		\includegraphics[width=0.85\linewidth]{imagens/regressailinear.png}\\
	\end{center}
	\caption[relação entre os dados reais e os previstos]{relação entre os dados reais e os previstos}
	\label{fig:logo}
	%%\legend{Fonte: Próprio Autor}
\end{figure}

O algoritmo de regressão linear com a base2\footnote[5]{base de dados de 2007 a 2019 com 30.000 jogos} teve uma acurácia de 68\% veja o grafico de relação entre os dados reais e os previstos. Após aplicar o \textit{Cross Validation} essa taxa teve um aumento de 3.6\%.
\begin{figure}[htbp]
	\begin{center}
		\includegraphics[width=0.85\linewidth]{imagens/regressailinearAPI.png}\\
	\end{center}
	\caption[relação entre os dados reais e os previstos]{relação entre os dados reais e os previstos}
	\label{fig:logo}
	%%\legend{Fonte: Próprio Autor}
\end{figure}

O algoritmo de arvore decisão com a base1\footnote[4]{base de dados de 2014 a 2018 com 9.840 jogo} teve uma acurácia de 89.7\% veja o grafico de relação entre os dados reais e os previstos. Após aplicar o \textit{Cross Validation} essa taxa teve um aumento de 4\%.
\begin{figure}[htbp]
	\begin{center}
		\includegraphics[width=1.2\linewidth]{imagens/arvoredecisao.png}\\
	\end{center}
	\caption[relação entre os dados reais e os previstos]{relação entre os dados reais e os previstos}
	\label{fig:logo}
	%%\legend{Fonte: Próprio Autor}
\end{figure}

O algoritmo de arvore decisão com a base2\footnote[5]{base de dados de 2007 a 2019 com 30.000 jogos} teve um taxa de acerto de 97\% veja o grafico de relação entre os dados reais e os previstos. Após aplicar o \textit{Cross Validation} essa taxa teve um  aumento de 3\%.
\begin{figure}[htbp]
	\begin{center}
		\includegraphics[width=1.2\linewidth]{imagens/arvoredecisaoAPI.png}\\
	\end{center}
	\caption[relação entre os dados reais e os previstos]{relação entre os dados reais e os previstos}
	\label{fig:logo}
	%%\legend{Fonte: Próprio Autor}
\end{figure}
\newpage
O algoritmo de k-NN\footnote[3]{k-nearest neighbors} com a base1 teve uma acurácia de 84.5\% veja o grafico de relação entre os dados reais e os previstos. Após aplicar o \textit{Cross Validation} essa taxa teve um aumento de 3.3\%.
\begin{figure}[htbp]
	\begin{center}
		\includegraphics[width=1.2\linewidth]{imagens/knn.png}\\
	\end{center}
	\caption[relação entre os dados reais e os previstos]{relação entre os dados reais e os previstos}
	\label{fig:logo}
	%%\legend{Fonte: Próprio Autor}
\end{figure}

O algoritmo de k-NN com a base2 teve uma acurácia de acerto de 87\% veja o grafico de relação entre os dados reais e os previstos.Após aplicar o \textit{Cross Validation} essa taxa teve um aumento de 3.3\%.
\begin{figure}[htbp]
	\begin{center}
		\includegraphics[width=1.2\linewidth]{imagens/knnAPI.png}\\
	\end{center}
	\caption[relação entre os dados reais e os previstos]{relação entre os dados reais e os previstos}
	\label{fig:logo}
	%%\legend{Fonte: Próprio Autor}
\end{figure}

\newpage
O algoritmo de floresta aleatória com a base2 inicialmente foi testado com 10 arvores chegando a um resultado de 91\%  e 94 \% com 25 arvores de acurácia.
\begin{figure}[htbp]
	\begin{center}
		\includegraphics[width=1.2\linewidth]{imagens/florestaaleatoria.png}\\
	\end{center}
	\caption[relação entre os dados reais e os previstos com 10 e 25 arvores]{relação entre os dados reais e os previstos com 10 e 25 arvores}
	\label{fig:logo}
	%%\legend{Fonte: Próprio Autor}
\end{figure}

O algoritmo de floresta aleatória com a base2 com 53 arvores conseguiu atingir uma faixa de 99\% e 100\% de acurácia. Para  ter um taxa contante de 100\% de acurácia são necessária mais de 100 arvores.
\begin{figure}[htbp]
	\begin{center}
		\includegraphics[width=1.2\linewidth]{imagens/florestaaleatoriaAPI.png}\\
	\end{center}
	\caption[relação entre os dados reais e os previstos com 53 e 120 arvores]{relação entre os dados reais e os previstos com 53 e 120 arvores}
	\label{fig:logo}
	%%\legend{Fonte: Próprio Autor}
\end{figure}
\newpage
O algoritmo de regressão logística tanto com a base1\footnote[4]{base de dados de 2014 a 2018 com 9.840 jogo} é com a  base2\footnote[5]{base de dados de 2007 a 2019 com 30.000 jogos} teve uma acurácia de 99\% a 100\% veja o gráfico de relação entre os dados reais e os previstos.
\begin{figure}[htbp]
	\begin{center}
		\includegraphics[width=1.2\linewidth]{imagens/regressaologistocaAPI.png}\\
	\end{center}
	\caption[relação entre os dados reais e os previstos]{relação entre os dados reais e os previstos}
	\label{fig:logo}
	%%\legend{Fonte: Próprio Autor}
\end{figure}

O algoritmo de máquinas de vetores de suporte com a base1 teve uma acurácia de 80\% veja o gráfico de relação entre os dados reais e os previstos.Após aplicar o \textit{Cross Validation} essa taxa teve um aumento de 2.7\%.

\begin{figure}[htbp]
	\begin{center}
		\includegraphics[width=1.2\linewidth]{imagens/SVM.png}\\
	\end{center}
	\caption[relação entre os dados reais e os previstos]{relação entre os dados reais e os previstos}
	\label{fig:logo}
	%%\legend{Fonte: Próprio Autor}
\end{figure}
\newpage
O algoritmo de máquinas de vetores de suporte com a base2\footnote[5]{base de dados de 2007 a 2019 com 30.000 jogos} teve uma acurácia de 87\%, veja o gráfico de relação entre os dados reais e os previstos.Após aplicar o \textit{Cross Validation} essa taxa teve um aumento de 3.7\%.
\begin{figure}[htbp]
	\begin{center}
		\includegraphics[width=1.2\linewidth]{imagens/SVMAPI.png}\\
	\end{center}
	\caption[relação entre os dados reais e os previstos]{relação entre os dados reais e os previstos}
	\label{fig:logo}
	%%\legend{Fonte: Próprio Autor}
\end{figure}


