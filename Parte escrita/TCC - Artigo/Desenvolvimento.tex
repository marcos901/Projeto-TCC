\section{DESENVOLVIMENTO}
\subsection{MATERIAIS}
Os materiais usados no trabalho foram duas bases de dados ambas da NBA Advanced Stats\footnote{National Basketball Associativo} sendo uma da \textit{season} de 2014 a 2018 com 9.840 jogos com os dados armazenados em um arquivo csv, a outra indo da \textit{season} de 2007 a 2019 com 30.000 jogos com os dados armazenados em um banco de dados e sendo acessado através da nba-api PyPI. 

A ferramenta utilizada para o desenvolvimento foi o \textit{JupyterLab}, linguagem de programação {python} e o uso das bibliotecas  pandas, numpy, \textit{sklearn}, \textit{seaborn}, matplotlib.

A NBA advances stats e um site patrocinado pela SAP com o proposito e de manter um registros de toda liga da NBA e facilitar e acesso a essa informações pelas equipes e organizações. A nba-api PyPI e uma API para acesso a www.nba.com, o principal objetivo e mapear e analisar o maior numero posivel de jogos.

O JupyterLab é um ambiente de desenvolvimento interativo baseado na Web para notebooks. O JupyterLab é facil de configurar e organizar a interface do usuário para suportar uma ampla variedade de fluxos de trabalho em ciência de dados, computação científica e aprendizado de máquina.O JupyterLab é extensível e modular e fácil de adicionar os plug-ins, que adicionam novos componentes e se integram aos já existentes.

Python é uma linguagem de programação criada por Guido van Rossum em 1991. Os objetivos do projeto da linguagem eram produtividade e legibilidade, é uma linguagem de alto nível, multi paradigma, suporta o paradigma orientado a objetos, imperativo, funcional e procedural. Possui tipagem dinâmica e uma de suas principais características é permitir a fácil leitura do código e exigir poucas linhas de código se comparado ao mesmo programa em outras linguagens.




\subsection{METODOLOGIA}
Os algoritmos usados para as predições são os de regressão linear, regressão logística, k Nearest Neighbors,  arvore de decisão, floresta aleatória,  Máquinas de Vetores de Suporte.

O algoritmo de regressão linear responsável por modelar uma associação entre uma ou mais variáveis de saída e entrada. O processo de regressão pode ser dividido em duas categorias, as paramétricas, no qual o relacionamento entre as variáveis é conhecido, e não paramétricas onde não existe conhecimento preexistente entre as variáveis. As técnicas de regressão linear procuram a relação entre duas variáveis por meio de uma equação de uma linha reta.
