%\thispagestyle{empty}
\newpage
\section{DISCUSSÃO E CONCLUSÃO} 
Conforme observado nos teste conclui-se que os algoritmos testados funciona melhor com dados de classe binaria,sendo assim dados como vitoria x derrota. 

O algoritmo de regressão linear que teve o pior despenho no teste, devido a construção dele trabalhar melhor com dados lineares. O que dado que foi utilizado consistia em prever derrota ou vitoria ou seja, o valor da predição era de forma binaria. Assim a técnicas de regressão linear procuram a relação entre duas variáveis por meio de uma equação de uma linha reta.

O algoritmos arvore de decisão, como funciona na forma de um fluxograma em pra suas tomadas de decisão vai depender da quantidade e qualidade dos dados com a qual essa arvore foi treinada. Como pode ser visto nos resultado a taxa de acertos com a utilização da base1\footnote[4]{base de dados de 2014 a 2018 com 9.840 jogo}, quando testado um a base2\footnote[5]{base de dados de 2007 a 2019 com 30.000 jogos} que possui mais que o dobre de dados a taxa teve um grande aumento.

O algoritmo k-NN\footnote[3]{k-nearest neighbors} trabalho com os vizinhos mais deve ser considerado como um método
no qual baseia-se por instâncias, isto é, ele vai determinar a classe de um objeto
desconhecido através da classe de outras instâncias.

Floresta aleatória são um conjunto de árvores de decisão trabalhando em conjunto, com um maior numero de arvore a taxa de predição também aumenta. Nos teste realizados com 10 árvores tinha uma aula porcentagem de acertos quando chegando acima de 53 arvore a taxa varia de 99\% a 100\%, mas quanto o maior numero de árvores o tempo do teste vai aumentando significativamente. Para ter uma taxa de constante de 100\% seria necessario mas 100 árvores isso exige um grande tempo.

A regressão logística foi o algoritmo com a maior taxa de acertos, pois ele trabalha com os fatores binário de predição,sendo assim o opostos da regressão linear. Como os teste foram feitos para prever a derrota e a vitoria.Assim a regressão logística calcula uma razão de probabilidade da variável alvo, que posteriormente é convertida em uma variável de base logarítmica, permitindo assim a classificação com base na aproximação de um dos valores. 

A máquinas de vetores trabalha definindo um limite linear logo para realizar a classificação ele separa os dados é os analisa para reconhecer padrões, assim que a uma entrada de um conjunto de dados e adicionada ele vai realizar a analise e dividir em duas classes, na qual as duas possíveis classes faz parte do classificador linear binário não probabilístico.

Podemos concluir que os algoritmos que obtiveram a melhor performance mais vezes foram
nesta ordem: regressão logica, floresta aleatória, arvore de decisão, knn, maquina de vetores de suporte,regressão linear. 

Todos os algoritmos abordados neste trabalho foram implementados em Python e encontramse dısponıveis em https://github.com/marcos901/Projeto-TCC/tree/master/Teste\%20dos\%20algoritmos\%20\%202019.





